\documentclass{article}

% packages
\usepackage{arxiv}
\usepackage[utf8]{inputenc} % allow utf-8 input
\usepackage[T1]{fontenc}    % use 8-bit T1 fonts
\usepackage{hyperref}       % hyperlinks
\usepackage{url}            % simple URL typesetting
\usepackage{booktabs}       % professional-quality tables
\usepackage{amsfonts}       % blackboard math symbols
\usepackage{nicefrac}       % compact symbols for 1/2, etc.
\usepackage{microtype}      % microtypography
\usepackage{lipsum}
\usepackage{siunitx}
\usepackage[linesnumbered,ruled,vlined]{algorithm2e}
\usepackage{algorithmic}
\usepackage{float}
\usepackage{amsmath}
\usepackage{xfrac}
\usepackage{enumerate}
\usepackage{cite}
\usepackage{xcolor}
\usepackage{caption}
\usepackage{graphicx}
\usepackage{subcaption}
\usepackage{wrapfig}


% general config
\graphicspath{ {./images/} }
\newcommand{\note}[1]{\textbf{#1}}
\newcommand{\todo}[1]{\textcolor{red}{#1}}
\newcommand{\tochoose}[1]{\begin{pmatrix} 21 \\ #1 \end{pmatrix}}

% algorithm2e commands
\SetKwComment{Comment}{/* }{ */}
\SetKwInOut{Input}{Input}
\SetKwInOut{Output}{Output}
\SetKwFor{RepTimes}{repeat}{times}{end}

% title and authors
\title{Genetic Algorithms and Evolution Strategies \\ for Neural Architecture Search}
\author{
    Josef Hamelink \\
    s2233827 \\
    \texttt{s2233827@umail.leidenuniv.nl} \\
    \And
    Koen van der Burg \\
     s2714892 \\
    \texttt{s2714892@umail.leidenuniv.nl} \\
}

\begin{document}

\maketitle


% ---------------------------------------------------------------- %
% -------------------------- ABSTRACT ---------------------------- %
% ---------------------------------------------------------------- %

\begin{abstract}
Neural networks, while powerful tools, can be challenging to design and train. 
Neural Architecture Search (NAS) is a method to automatically find the optimal neural network architecture for a given task, but the search space for NAS is large.
In this paper, we present a Genetic Algorithm (GA) and an Evolution Strategy (ES) to efficiently approach the NAS problem.
Multiple experiments have been conducted to find the optimal parameters for both algorithms.
The results show that GA is more effective for this particular problem.
When collated to ES, it is more fit to handle discrete problems, such as NAS in the way it is formulated in this paper.
\end{abstract}


% ---------------------------------------------------------------- %
% ---------------------------- INTRO ----------------------------- %
% ---------------------------------------------------------------- %

\section{Introduction}
\label{sec:intro}

For the practical assignment of the \textit{Evolutionary Algorithms} course (2022), we have been tasked with the implementation of a Genetic Algorithm (GA) and an Evolution Strategy (ES) to tackle a \textit{Neural Architecture Search} (NAS) problem.
While neural networks provide an extremely powerful way to solve a plethora of high-level tasks such as image classification, object detection and speech recognition, they are also complex to design.
Intricate architectures can require tremendous time and resources to train.
NAS is a method to automatically find the optimal neural network architecture for a given task.
\textit{NAS-Bench-101} \cite{nasbench101} is a project that provides a dataset with the performance of over 400 thousand convolutional neural network architectures. 
The bench-marker is trained on the CIFAR-10 dataset \cite{cifar10}; a widely used benchmark for image classification.

The goal of this assignment is to implement a GA and an ES that can efficiently search the space of these neural network architectures to find a good solution.
Genetic Algorithms are generally applied to discrete problems, while Evolution Strategies are more suited for continuous (real-valued) problems.
The NAS problem is inherently discrete, as it is the task of finding the optimal architecture for a neural network, which is described as a set of discrete variables.
This means that applying a Evolution Strategy approach will require an encoding step, to handle the aforementioned continuous values.

In order to evaluate the performance of our implementations of the evolutionary algorithms, we use the \textit{IOHexperimenter} and \textit{IOHanalyzer} components of the IOH platform, which was built by researchers at Leiden University \cite{IOH}.
To support readability of the source code that has been developed for this project, a workflow schema is provided.
This figure can be found in the appendix (see \ref{app:workflow}).


% ---------------------------------------------------------------- %
% ---------------------------- PROBLEM --------------------------- %
% ---------------------------------------------------------------- %

\section{Problem Description}
\label{sec:prob}

As mentioned before, NAS-bench-101 is utilized to test the neural networks performance.
Since this database only has the exact results of over 400 thousand architectures, constrictions to these networks have to be made.
NAS-bench-101 restricts the search of neural network configurations to relatively small, feed-forward only, structures.
These small structures are often labeled ``cells''.
These cells are integrated into the larger NAS-bench architecture.
This architecture is initialized with a convolutional stem layer, consisting of one 3x3 convolution with 128 output channels.
This data is fed into the first of three stacks.
Each stack holds 3 cells, followed by a down-sample layer.
The down sampling layer uses max-pooling to halve the images height and width, concurrently doubling the channel count.
After passing through 3 stacks and 2 down-sample layers, a final global average pooling layer is applied, as well as a dense (softmax) layer \cite{nas}.

Each cell has the same structure, which can be constructed from a space of different directed a-cyclic graphs.
In order to limit the size of the space, allowing exhaustive enumeration, some constrains are applied.
Each structure is limited to a maximum of 7 nodes, with two of them being labelled as the IN and OUT.
The remaining 5 nodes can be used for 3 different operations; a 3x3 convolution, a 1x1 convolution or a 3x3 max-pool (MP).
Limiting the amount of nodes also caps the maximum number of possible edges at 9.

For this specific assignment the encoding of the NAS problem is replaced by a set of discrete variables $x_i, i \in [1, 26]$.
Where the values $x_i, i \in [1, 21]$ are either 0 or 1, representing the upper-triangular binary matrix of NAS problem.
This triangle of values represent the connected nodes of the NAS bench cell architecture.
The latter values $x_i \in [22, 26]$ are either 0,1 or 2, representing the three different operations (3x3 conv, 1x1 conv and 3x3 MP respectively) possible on the 5 nodes of the aforementioned cell.
In order to calculate the fitness value of a candidate solution $x$ ($\{x_1, \dots, x_26\}$), the accuracy rate from the NAS bench-marking is used, which ranges from 0 to 1.
The accuracy rates are defined as the proportion of correctly classified images in the test set after the network has trained for 108 epochs.

A candidate solution $x = \{x_1, \dots , x_{26}\}$ maps to an input of NAS-bench as; 

$ \begin{bmatrix}
0 & x_1 & x_2 & x_3    & x_4    & x_5    & x_6    \\
0 & 0   & x_7 & x_8    & x_9    & x_{10} & x_{11} \\
0 & 0   & 0   & x_{12} & x_{13} & x_{14} & x_{15} \\
0 & 0   & 0   & 0      & x_{16} & x_{17} & x_{18} \\
0 & 0   & 0   & 0      & 0      & x_{19} & x_{20} \\
0 & 0   & 0   & 0      & 0      & 0      & x_{21} \\
0 & 0   & 0   & 0      & 0      & 0      & 0 
\end{bmatrix}  $

And the operations at the 7 different nodes = [input, $x_{22}$, $x_{23}$, $x_{24}$, $x_{25}$, $x_{26}$, output]


% ---------------------------------------------------------------- %
% ------------------------------ GA ------------------------------ %
% ---------------------------------------------------------------- %

\section{Genetic Algorithm}
\label{sec:GA}

Now that the general problem is understood properly, a more in depth description and solution can be discussed.
Formally, the function $f: \mathbb{Z}^n \rightarrow \mathbb{R}$ is to be maximized.
The problem can be defined as (for $n=26$):

\begin{itemize}
    \item A candidate solution is noted as $x$, with
    \begin{itemize}
        \item $x_{i, i \in [1, \dots, 21]} \in \{0, 1\}$; and
        \item $x_{i, i \in [22, \dots, 26]} \in \{0, 1, 2\}$
    \end{itemize}
    \item It has a fitness (function) value $f(x) \in [0, 1] \; \forall \; x \in \mathbb{Z}^{26}$
    \item The optimal function value is defined as $f^* = f(x^*)$, where $f(x^*) \geq f(x_{\mathrm{observed}}) \; \forall \; x_{\mathrm{observed}} \in \mathbb{Z}^{26}$
    \item An invalid\footnote{
        An invalid solution is a solution that does not map to a valid neural architecture. 
        This can be caused by a number of reasons, such as a non-connected graph, or a graph with a cycle.
        } solution $x_{\mathrm{invalid}}$ maps to $f(x_{\mathrm{invalid}})=0$
\end{itemize}

The key idea of a genetic algorithm is to use a population of candidate solutions to find the optimal solution.
At its very core, a genetic algorithm is a search algorithm that mimics the process of natural selection; using the best solutions to create new solutions.
The main steps of a genetic algorithm are as follows:
Initialize a population of \texttt{N} random candidate solutions.
From the population, select a number of ($\mu$) parents based on their fitness values.
These parents are then used to create a number of ($\lambda$) offspring by applying a recombination method on two randomly selected parents.
The offspring are mutated, and the new population can either be just the offspring ($\lambda$) or the parents and offspring combined ($\mu+\lambda$).
The process is repeated until a stopping criterion is met, in our case when the budget is exhausted, making the algorithm a \textit{fixed budget} algorithm.
A more formal description of this process is given in Algorithm \ref{alg:GA}.

% --------------------------- ALGORITHM -------------------------- %

\begin{algorithm}[htbp]
    
    \caption{Genetic Algorithm}
    \label{alg:GA}
    \SetAlgoLined
    \DontPrintSemicolon
    
    \Input{
        $f$,
        \texttt{N},
        $\lambda$,
        $\mu$,
        \texttt{budget},
        \texttt{recombination},
        \texttt{selection},
        \texttt{mutation}
    }
    \Output{$x^*$, $f^*$}

    \BlankLine

    % define x_opt & f_opt
    $x^* \gets$ \texttt{None}\;
    $f^* \gets - \infty$;

    \BlankLine

    % initialize population
    $\mathbf{X} \gets$ [] \Comment*[r]{$\mathbf{X}$ will contain the population of solutions}
    \RepTimes{\texttt{N}}{
        $x_{i, i \in [1, \dots, 21]} \gets U(\{0, 1\})$ \Comment*[r]{matrix part (binary)}
        $x_{i, i \in [22, \dots, 26]} \gets U(\{0, 1, 2\})$ \Comment*[r]{operations part (ternary)}
        append $x$ to $\mathbf{X}$\;
    }

    \BlankLine

    % main loop
    \While{\texttt{budget}}{
        
        % evaluate population
        $\mathbf{F}$ $\gets$ [$f(x)$ for $x$ in $\mathbf{X}$] \Comment*[r]{evaluate population with fitness function $f$}
        \texttt{budget} $\gets$ \texttt{budget} - \texttt{N}\;
        
        % update x_opt & f_opt
        \If{\texttt{max} ($\mathbf{F}$) $> f^*$}{
            $f^* \gets$ \texttt{max}($\mathbf{F}$)\;
            $x^* \gets$ $\mathbf{X}$[\texttt{argmax}($\mathbf{F}$)] \Comment*[r]{if better fitness value found, update best solution}
        }

        % evolutionary operators
        \texttt{parents} $\gets$ \texttt{selection}($\mathbf{X}$, $\mathbf{F}$, $\mu$) \Comment*[r]{select $\mu$ parents}
        \texttt{offspring} $\gets$ \texttt{recombination}(\texttt{parents}, $\lambda$) \Comment*[r]{produce $\lambda$ offspring}
        \texttt{offspring} $\gets$ \texttt{mutation}(\texttt{offspring}) \Comment*[r]{mutate offspring}

        % update population for next iteration
        \eIf{\texttt{N} $=\lambda$}{
            $\mathbf{X} \gets$ \texttt{offspring}\;
        }{
            $\mathbf{X} \gets$ \texttt{parents} + \texttt{offspring}\;
        }
    }
    \BlankLine
    \Return{$x^*, f^*$}

\end{algorithm}

\newpage

 % ---------------------------- PARAMS (overview) ---------------- %

In our experiments, the parameters $f$ (objective function) and \texttt{budget} for this algorithm are fixed, while the remaining parameters are varied.
In order to find the best configuration of variables and operators for our genetic algorithm, we performed a grid search over the following parameters:

\begin{minipage}{0.4\textwidth}   
    \begin{itemize}
        \item \texttt{N}, $\mu$, $\lambda$: population composition
        \item \texttt{selection}: selection method
        \begin{itemize}
            \item roulette wheel selection
            \item tournament selection
            \item rank selection
            \item stochastic universal sampling
        \end{itemize}
    \end{itemize}
\end{minipage}
~
\begin{minipage}{0.55\textwidth}
    \begin{itemize}
        \item \texttt{recombination}: recombination method
        \begin{itemize}
            \item $k$-point crossover
            \item uniform crossover
        \end{itemize}
        \item \texttt{mutation}: mutation method
        \begin{itemize}
            \item bit-flip mutation (always flipping $n_{\mathrm{bits}}$ bits)
            \item uniform mutation (bit flips with probability $p$)
        \end{itemize}
    \end{itemize}
\end{minipage}

% --------------------------- PARAMS (descriptions) -------------- %

\subsection{Population Composition}
\label{subsec:ga_pop}

The population composition is defined by the parameters \texttt{N}, $\mu$, and $\lambda$.
The population size (\texttt{N}) is the number of individuals in the population; in the search for optimal parameters, values of $100$ and $40$ are tested.
The number of parents ($\mu$) is the number of individuals that are selected to produce offspring.
For this parameter, examinations of $20$ and $40$ with $\texttt{N}=100$, and $10$ and $20$ with $\texttt{N}=40$, have been realized.
The number of offspring ($\lambda$) is the number of individuals that are produced by the recombination operator.
For each setting of $\mu$, two values can be used for $\lambda$, e.g. when $\texttt{N}=100$ and $\mu=20$, both $80$ and $100$ are valid choices.

It can be appreciated that some configurations will result in what is commonly referred to as $\mu + \lambda$ (or \textit{elitist}) algorithms; where the population size is equal to the sum of number of parents and offspring.
Other configurations will result in $\mu, \lambda$-evolutionary algorithms, where the number of offspring equals the population size, meaning that no parents will be carried over to the next generation.


\subsection{Selection Methods}
\label{subsec:ga_sel}

In genetic algorithms, multiple distinct selection methods exist, as simply selecting the best individuals from the population often leads to premature convergence to local optima.
While there are many different selection methods, we have chosen to focus on three of the most common ones: roulette wheel selection, tournament selection and rank selection.
Extra study has been done on implementing a less prevalent selection method; stochastic universal sampling (SUS).

\subsubsection*{Roulette Wheel Selection}
\label{subsubsec:ga_sel_rws}
Roulette wheel selection \cite{roulettewheel}, also known as fitness proportionate selection (FPS), is a method where the probability of an individual being selected is directly proportional to its fitness.
The method can be imagined as a roulette wheel, where the fitness of each individual is represented by a slice of the wheel, with the size of the slice proportional to the fitness of the individual.
The wheel is then spun, and the individual that the wheel lands on is selected.
In this paper's implementation, an individual is not ``removed'' from the roulette wheel after having been selected, which leads to the possibility of an individual being selected multiple times.

\subsubsection*{Tournament Selection}
\label{subsubsec:ga_sel_ts}
\vspace{-0.2cm}
Tournament selection \cite{tournament} is a method where $k$ individuals are selected at random from the population, and the individual with the highest fitness is selected to be part of the parents.
The parameter $k$ is referred to as the tournament size, and is a tune-able hyperparameter.
In order to account for the \textit{curse of dimensionality} when it comes to the grid-search over parameter configurations,tournament size is fixed to $k=2$.

\subsubsection*{Rank Selection}
\label{subsubsec:ga_sel_rs}
\vspace{-0.2cm}
Rank selection \cite{rank} is quite similar to roulette wheel selection, but instead of using the fitness of the individuals as the size of the slice, the rank of the individual is used.
After sorting the population based on their fitness values, the individuals are assigned a rank, where the individual with the highest fitness is assigned rank 1.
The probability distribution for a given population size \texttt{N} can be described as:

$$ P(\texttt{N}) = \begin{bmatrix} \texttt{N} \\ \texttt{N}-1 \\ \vdots \\ 1 \end{bmatrix} * \frac{2}{\texttt{N}(\texttt{N}+1)}$$

For example, a population size $\texttt{N}=3$ always results in a probability distribution of
$P(\texttt{N}) = \begin{bmatrix} \sfrac{1}{2}, \sfrac{1}{3}, \sfrac{1}{6} \end{bmatrix}$.
The justification for using rank selection over roulette wheel selection is that, when a point has been reached where fitness values of all individual solutions are sufficiently similar, roulette wheel selection will be reduced to random selection.
This is undesirable, as the evolutionary pressure to attain high fitness is then lost.

\subsubsection*{Stochastic Universal Sampling}
\label{subsubsec:ga_sel_sus}
\vspace{-0.2cm}
The final selection method we have implemented, stochastic universal sampling (SUS) \cite{sus}, is another derivative of fitness proportional selection.
The previously mentioned roulette wheel selection is slightly modified, where the same roulette wheel is used, with the same ratios.
The difference is the use of numerous selection points that are evenly distributed around the wheel rather than utilizing a single selection point and rotating the roulette wheel again until all required individuals have been selected.
In this manner, everyone is chosen simultaneously.
This method gives opportunity to weaker individuals of the population, and prevents oversaturation of the fittest members.


\newpage


\subsection{Recombination Methods}
\label{subsec:ga_rec}

After selecting the parents that are allowed -- but not \textit{guaranteed} -- to produce offspring, recombination occurs to introduce variance.
Two different recombination methods have been implemented; $k$-point crossover and uniform recombination.
Both methods sample two random parents from the selected parents, and produce a child from the two.

\subsubsection*{K-Point Crossover}
\label{subsubsec:ga_rec_kp}
\vspace{-0.2cm}
$k$-point crossover introduces a fixed number ($k$) of ``splits'' along the genomes of parents 1 and 2.
Part of the solution between each 2 points is swapped between parent 1 and parent 2, creating novel solutions.
Both resulting children are added to the population, ensuring that no ``good'' variables are lost.
Increasing $k$ causes more slices to be swapped, and thus more variance in the offspring.
A hyperparameter that can be adjusted in the current implementation is $k$. 
The variables $k=1, 2, 3$ are the focus of our investigation.

\subsubsection*{Uniform Recombination}
\label{subsubsec:ga_rec_uc}
\vspace{-0.2cm}
The second method, uniform crossover, can be seen as similar to $k$-point crossover, because it also produces two children from two parents.
The distinction is that each bit in parent 1 $\begin{bmatrix} x_1, x_2, \dots, x_26 \end{bmatrix}$ is randomly assigned to either child 1 or child 2, with equal probability.
The other child is then given the bit from parent 2 that is located in the same location.
Uniform crossover is different from $k$-point crossover with $k$ set to $n-1$ ($n$ is the length of the bitstring), because it allows for multiple consecutive bits from the same parent to be assigned to a child, whereas in $n-1$-point crossover, this would never be the case.

\subsection{Mutation Methods}
\label{subsec:ga_mut}

When selection and recombination have taken place, mutations are introduced, to further increase small variance when compared to the original solution set. 
Compared to it's biological part, they operate similarly, and have the same function; maintaining genetic diversity. 
By avoiding the population of solutions from becoming overly similar to one another, mutation operators try to avoid local minima, which could slow or even block convergence towards the global optimum.
This justification also motivates most GA systems to choose a random (or semi-random) group with a weighting toward the fitter individuals rather than just selecting the population's fittest members to create the subsequent generation.


\subsubsection*{Bitflip Mutation}
\label{subsubsec:ga_mut_bf}
\vspace{-0.2cm}
Bitflip mutation is a simple mutation method where a fixed number ($n_{\mathrm{bits}}$) of random bits are flipped, i.e. set to the opposite value.
When working with binary strings, this is equivalent to flipping a single bit from 0 to 1, or vice versa.
For the ternary part of our solution strings, the bitflip mutation will be defined as a random choice between the two other possible values, i.e. $ x_i = 2 \rightarrow p(x_{i,\mathrm{mut}} = 0) = p(x_{i, \mathrm{mut}} = 1) = \sfrac{1}{2}$.
The number of bits to be flipped is a hyperparameter that can be tuned.
In the papers experiments, the values $n_{\mathrm{bits}}=1, 2, 3$ are checked.

\subsubsection*{Uniform Mutation}
\label{subsubsec:ga_mut_um}
\vspace{-0.2cm}
Uniform mutation also flips bits in the solution string, but instead of flipping a fixed number of bits, each bit is flipped with a fixed probability ($p_{\mathrm{mut}}$).
This means that it is possible for a solution to not be mutated at all, or for all bits to be flipped.
The bitflip probability is generally set to a low value, often $\sfrac{1}{n}$, where $n$ is the length of the solution string.
For the NAS problem, we get $n=26$, so $p_{\mathrm{mut}} = \sfrac{1}{26} \approx 0.04$.
To evaluate whether this approach holds for our specific problem, other mutation rates $0.1$ and $0.01$ are tried.


% ---------------------------------------------------------------- %
% ------------------------------ ES ------------------------------ %
% ---------------------------------------------------------------- %

\section{Evolution Strategy}
\label{sec:ES}
Evolution strategies differ from genetic algorithms in that they can operate in the continuous domain, rather than the discrete domain.
Whereas genetic algorithms are sometimes still used for continues problems, they require an additional step to discretize the solutions; this is called \textit{encoding} or \textit{decoding}, depending on the direction of the transformation.
However, for this problem, we are presented with an objective function that takes discrete solutions.
This means, that we must reverse what would normally be done to adapt a GA to a continuous problem, in order to adapt our ES to a discrete problem.

Evolution strategies have another advantage over genetic algorithms; they can employ what is called \textit{self-adaptation}.
Meaning, the algorithm can adapt its own parameters during the search process.
This is done by using two variables; the \textit{step-size} $\sigma$ (also referred to as mutation strength) and the \textit{learning rate} $\tau$.
The step-size is used to determine the strength of the mutation, and the learning rate in turn controls the step-size; decreasing it as the algorithm progresses towards an optimum.

\newpage

Evolution strategy algorithms come in many different flavors \cite{CMA-ES}\cite{NES}, but in order to compare them to genetic algorithms, we will use largely the same setup as described in Section~\ref{sec:GA}.
Again, we start by initializing a population of \texttt{N} random candidate solutions, this time with a continuous representation.
Along with every solution $x$, one or more step-sizes ($\sigma_x$) are initialized based on the size of the search space and a constant $\sigma_0$.
From this population, we select a number of ($\mu$) parents based on their fitness values.
In contrast to the genetic algorithm, this selection is entirely deterministic, i.e. only the ``best'' $\mu$ solutions are selected.
Recombination is then performed on the selected parents, and the resulting children are mutated by adding a random value to each parameter, drawn from a normal distribution with mean 0 and standard deviation $\sigma$.
The new population can then again be created through either $\mu+\lambda$, or $\mu,\lambda$ selection.
An overview of the algorithm is given in Algorithm~\ref{alg:ES}, with the encoding and mutation steps being separated into their own respective algorithms \ref{alg:ES_enc} and \ref{alg:ES_mut} for brevity.
The variables $f$ and \texttt{budget} are again fixed, and the remaining input parameters are varied in our experiments.

% --- base alg --- %

\begin{algorithm}[htbp]

    \caption{Evolution Strategy}
    \label{alg:ES}
    \SetAlgoLined
    \DontPrintSemicolon

    \Input{
        $f$,
        \texttt{N},
        $\lambda$,
        $\mu$,
        $\sigma_0$,
        $\tau$,
        \texttt{budget},
        \texttt{recombination},
        \texttt{chunk\_size}
    }
    \Output{$x^*$, $f^*$}

    \BlankLine

    % define x_opt & f_opt
    $x^* \gets$ \texttt{None}\;
    $f^* \gets - \infty$;
    
    \BlankLine

    % initialize population
    \texttt{lb\_mat}, \texttt{lb\_ops} $\gets$ 0, 0 \Comment*[r]{lower bounds}
    \texttt{ub\_mat} $\gets$ $2^\texttt{chunk\_size}-1$ \Comment*[r]{upper bounds are dependent on chunk size}
    \texttt{ub\_ops} $\gets$ $3^5-1$ \Comment*[r]{upper bounds, "chunk size" is always 5}
    \texttt{ops\_idx} $\gets 21 / \texttt{chunk\_size} + 1$ \Comment*[r]{define index of operations gene}
    $\mathbf{X} \gets$ [] \Comment*[r]{$\mathbf{X}$ will contain the solutions, along with their sigmas}
    \RepTimes{\texttt{N}}{
        $x_{i, \: i \in \{1, \; \dots, \; \texttt{ops\_idx}-1\}} \gets U([\texttt{lb\_mat}, \texttt{ub\_mat}])$
            \Comment*[r]{matrix genes ($\mathbb{R}$)}
        $x_{i, \: i = \texttt{ops\_idx}} \gets U([\texttt{lb\_ops}, \texttt{ub\_ops}])$
            \Comment*[r]{operations gene ($\mathbb{R}$)}
        $\sigma_{x, \: \texttt{mat}} \gets \sigma_0 * U([\texttt{lb\_mat}, \texttt{ub\_mat}])$
            \Comment*[r]{initialize $\sigma$ for matrix part ($\mathbb{R}$)}
        $\sigma_{x, \: \texttt{ops}} \gets \sigma_0 * U([\texttt{lb\_ops}, \texttt{ub\_ops}])$
            \Comment*[r]{initialize $\sigma$ for operations part ($\mathbb{R}$)}
        append $(x, \sigma_x)$ to $\mathbf{X}$\;
    }

    \BlankLine

    % main loop
    \While{\texttt{budget}}{
        
        % evaluate population
        $\mathbf{X}_{\texttt{encoded}}$ $\gets$ [$\texttt{encode}(x)$ for $x, \_$ in $\mathbf{X}$]
            \Comment*[r]{encode population to binary \& ternary}
        $\mathbf{F}$ $\gets$ [$f(x_\texttt{e})$ for $x_\texttt{e}$ in $\mathbf{X}_{\texttt{encoded}}$]
            \Comment*[r]{evaluate population with fitness function $f$}
        \texttt{budget} $\gets$ \texttt{budget} - \texttt{N}\;
        % update x_opt & f_opt
        \If{\texttt{max} ($\mathbf{F}$) $> f^*$}{
            $f^* \gets$ \texttt{max}($\mathbf{F}$)\;
            \_, $x^* \gets$ $\mathbf{X}$[\texttt{argmax}($\mathbf{F}$)]
                \Comment*[r]{if better fitness value found, update best solution}
        }

        % evolutionary operators
        \texttt{parents} $\gets$ best $\mu$ of $\mathbf{X}$ based on $\mathbf{F}$
            \Comment*[r]{select $\mu$ parents}
        \texttt{offspring} $\gets$ \texttt{recombination}(\texttt{parents}, $\lambda$)
            \Comment*[r]{produce $\lambda$ offspring}
        \texttt{offspring} $\gets$ \texttt{mutation}(\texttt{offspring}, $\sigma_{\texttt{offspring}}$, $\tau$)
            \Comment*[r]{mutate offspring}

        % update population for next iteration
        \eIf{\texttt{N} $=\lambda$}{
            $\mathbf{X} \gets$ \texttt{offspring}\;
        }{
            $\mathbf{X} \gets$ \texttt{parents} + \texttt{offspring}\;
        }
    }
    \BlankLine
    \Return{$x^*, f^*$}

\end{algorithm}

% ---------------------------- PARAMS (overview) ---------------- %

\begin{minipage}{0.5\textwidth}   
    \begin{itemize}
        \item \texttt{N}, $\mu$, $\lambda$: population composition
        \item $\sigma$: initial mutation strength \\ (proportional to the search space)
        \item $\tau$: mutation strength decay
        \item \texttt{chunk\_size}: size of the discrete chunks \\ that each gene represents
    \end{itemize}
\end{minipage}
~
\begin{minipage}{0.45\textwidth}
    \begin{itemize}
        \item \texttt{recombination}: recombination method
        \begin{itemize}
            \item discrete recombination
            \item global discrete recombination
            \item intermediate recombination
            \item global intermediate recombination
        \end{itemize}
    \end{itemize}
\end{minipage}

\subsection{Encoding}
\label{sec:ES_enc}
\vspace{-0.2cm}
To tackle the problem of encoding the real-valued genes into binary or ternary, we use a simple approach.
We first define the index of the operations gene \texttt{ops\_idx} as the index of the first (and only) gene that represents the operations part of a solution.
Then, we loop over the genes of the matrix part, and encode each gene into a binary string.
The length of the string is determined by the \texttt{chunk\_size} parameter, which is the number of bits that each gene represents.
This parameter also determines the lower and upper bounds of the genes, since the number of possible values is $2^{\texttt{chunk\_size}}$.
For conceptual simplicity, we assume that the lower bound is 0, and the upper bound is $2^{\texttt{chunk\_size}}-1$.
Each real valued gene can then be rounded to the nearest integer, and converted to a binary string, which will be zero-padded to the length of \texttt{chunk\_size}.
A procedural overview of the encoding is shown in Algorithm~\ref{alg:ES_enc}.
No decoding is needed since the original real values are used for the next generation.

While theoretically possible, we decided it would be infeasible within the scope of this project to explore any other chunk sizes than $3$ and $7$, as they are strict divisors of $21$; the number of variables in the matrix part.
This allowed us to enforce the same bounds on all genes (in the matrix part), and as an extension of this, not have to worry about different mutation strengths.
For the operations part of the solution string, we map a single real valued gene to a ternary string of length $5$.

% --- encoding alg --- %
\vspace{-0.3cm}
\begin{algorithm}[htbp]

    \caption{Evolution Strategy Encoding}
    \label{alg:ES_enc}
    \SetAlgoLined
    \DontPrintSemicolon

    \Input{$x$, (\texttt{ops\_idx}, \texttt{chunk\_size})$^*$}
    \Output{$x_\texttt{e}$}

    \BlankLine

    % loop over matrix part
    $x_{\texttt{e}, \texttt{mat}} \gets$ []\;
    \For{$i$ in $[1, \; \dots, \; \texttt{ops\_idx}-1]$}{
        \Comment*[l]{e.g., 19 $\rightarrow$ 0010011 (with chunk\_size 7, so each gene ($x_i$) represents 7 bits)}
        \texttt{bin\_repr} $\gets$ \texttt{to\_binary}(\texttt{round}($x_i$))\;
        append \texttt{bin\_repr} to $x_\texttt{e}$\;
    }

    \Comment*[l]{e.g., 73 $\rightarrow$ 02201 (chunk\_size is 5 always 5 for operations gene)}
    \texttt{ter\_repr} $\gets$ \texttt{to\_ternary}(\texttt{round}($x_{\texttt{ops\_idx}}$))\;
    append \texttt{ter\_repr} to $x_\texttt{e}$\;

    \BlankLine

    \Return{$x$, $\sigma_x$}

\end{algorithm}

\vspace{-0.5cm}
\subsection{Mutation}
\label{sec:ES_mut}
\vspace{-0.2cm}
The mutation operator is the most distinctive part of the ES algorithm, as it is responsible for the exploration of the search space.
The mutation on the real-valued genes is performed by adding a random value to each gene, which is drawn from a normal distribution with mean 0 and standard deviation 1 and multiplied by the mutation strength (i.e., $\sigma * N(0, 1)$).
It is important to note that in traditional ESs, a choice can be made to use individual $\sigma$s for each gene, or a single $\sigma$ for all genes.
In the case of this problem, this is possible, but not desirable, as the search space of the genes in the operations part is considerably larger than the search space of the genes in the matrix part.
$3^5$ is $243$, while $2^7$ is $128$, and $2^3$ is just $8$, hence in Algorithm~\ref{alg:ES_mut} it can be observed that all mutation operations are split into two parts.

Because exploring the effect of using individual $\sigma$s versus a single $\sigma$ is still relevant, we define a ``single-$\sigma$'' approach as using the same mutation strength for all genes in the matrix part, and a ``multi-$\sigma$'' approach as using a different mutation strength for each gene in the matrix part.
The operations part will always use its own mutation strength.
To ensure that no invalid genes will arise during encoding, the genes are clipped to the bounds of the search space after mutation.
Lastly, the mutation strength itself is updated in accordance to the formula as recommended by H.P. Schwefel \cite{tau}.

% --- mutation alg --- %
\vspace{-0.3cm}
\begin{algorithm}[htbp]

    \caption{Evolution Strategy Mutation}
    \label{alg:ES_mut}
    \SetAlgoLined
    \DontPrintSemicolon

    \Input{$x$, $\sigma_x$, $\tau$, (\texttt{lb\_mat}, \texttt{ub\_mat}, \texttt{lb\_ops}, \texttt{ub\_ops})$^*$}
    \Output{$x$, $\sigma_x$}

    \BlankLine

    % mutate x
    $x_\texttt{mat} \gets$ $x + \sigma_{x, \texttt{mat}} * N(0, 1)$ \Comment*[r]{mutate $x$}
    $x_\texttt{ops} \gets$ $x + \sigma_{x, \texttt{ops}} * N(0, 1)$\;

    \BlankLine
    
    % clip x
    $x_\texttt{mat} \gets$ clip($x_\texttt{mat}$, $\texttt{lb\_mat}$, $\texttt{ub\_mat}$) \Comment*[r]{clip $x$}
    $x_\texttt{ops} \gets$ clip($x_\texttt{ops}$, $\texttt{lb\_ops}$, $\texttt{ub\_ops}$)\;
    
    \BlankLine

    % update sigma
    $\sigma_{x, \texttt{mat}} \gets$ $\sigma_{x, \texttt{mat}} * \exp{({\tau * N(0, 1)})}$ \Comment*[r]{mutate $\sigma_x$}
    $\sigma_{x, \texttt{ops}} \gets$ $\sigma_{x, \texttt{ops}} * \exp{({\tau * N(0, 1)})}$\;

    \BlankLine

    \Return{$x$, $\sigma_x$}

\end{algorithm}

\subsection{Recombination}
\label{sec:ES_rec}

The continuous nature of the ES algorithm makes it possible to use some recombination operators that are not possible in GA algorithms.
Due to these differences, our recombination operator does not create two children from two parents, but instead creates one child from two or more parents.
The four different recombination techniques are covered are: discrete and intermediate recombination, and their global variants.

Discrete recombination chooses two parent solutions randomly from the pool of selected parents, and for each gene, chooses one of the two parents from which to inherit the gene.
This is essentially identical to the \textit{uniform recombination} operator in GA algorithms.
Its global counterpart does not choose from just two parents, but for each parameter considers all the available parents, each having equal chance of being selected.

Intermediate recombination is very straightforward; it takes two parents and averages the parameters between the parents.
The resulting average parameters is the recombined child.
The global variant averages the parameters over all the parents.
It might be concerning that this method results in $\lambda$ identical children, but this is not a problem, as the mutation operator will introduce variety into the population shortly after the recombination step.


\section{Methodology}
\label{sec:meth}

As mentioned in Section~\ref{sec:prob}, the problem is the task of finding a set of parameters that will result in the best possible performance of the algorithm.
The total number of possible solutions given the constraints is $2^{21} \cdot 3^5 \approx 5.1 \cdot 10^8$, which is too large to be searched exhaustively.
However, we can immediately discard a considerable portion of the search space by using the fact that the matrix part of the solution can take a maximum of nine positive bits.
This prunes our search space to:

$$\left[ \tochoose{9} + \tochoose{8} + \dots + \tochoose{1}\right]\cdot 3^5 \approx 2.0 \cdot 10^8 $$

While large, it is feasible to search this space exhaustively, when considering the fact that each function evaluation is essentially a lookup operation, taking roughly $70$ nanoseconds on a modern CPU.
The time needed for such a brute-force search is \textasciitilde $4$ hours, which is acceptable for the purpose of this paper.
The number of configurations could be further reduced by using a more sophisticated pruning method to discard solutions that would result in different types of invalid neural architectures, but this is out of scope for this project.

When the search space has been mapped out, we start to focus on the different algorithms that can be used to search it more efficiently.
Before we can show a detailed comparison of the different algorithms, we need to define the parameters that will be used for the experiments.
To do this, we defined a handful of different values for all variable parameters and tried all possible combinations of these parameters.
The set of configurations that have been explored for the GA algorithm is the product of the parameters in Table~\ref{tab:GA_params}.
The notation ``\texttt{N}, $\mu$, ($\lambda_{1}$ \& $\lambda_{2}$)'' means that for a population size of \texttt{N}, with a number of parents $\mu$ and the number of children $\lambda$, both $\lambda_{1}$ and $\lambda_{2}$ have been tried for the \textit{plus} and \textit{comma} selection, respectively.
The total number of resulting configurations is then $8 \cdot 4 \cdot 4 \cdot 6 = 768$.
Each configuration is run $20$ individual times to reduce the effect of random noise.
Because we are just interested in results with a budget fixed to 5000 function evaluations, this is not too computationally expensive, only taking \textasciitilde $2\sfrac{1}{2}$ hours on a MacBook Air M1.

\begin{table}[htbp]
    \centering
    \begin{tabular}{llll}
        \toprule
        \texttt{N}, $\mu$, $\lambda$ & \texttt{selection}   & \texttt{recombination} & \texttt{mutation} \\
        \midrule
        100, 40, (60 \& 100)         & roulette wheel       & $k$-point ($k=1$)      & uniform (0.01)    \\
        100, 20, (80 \& 100)         & tournament           & $k$-point ($k=2$)      & uniform (0.04)    \\
        40, 20, (20 \& 40)           & rank                 & $k$-point ($k=3$)      & uniform (0.1)     \\
        40, 10, (30 \& 40)           & stochastic universal & uniform                & bit-flip (1)      \\
        -                            & -                    & -                      & bit-flip (2)      \\
        -                            & -                    & -                      & bit-flip (3)      \\
        \bottomrule
    \end{tabular}
    \vspace{0.1cm}
    \caption{The different parameters that have been tried for the GA algorithm.}
    \label{tab:GA_params}
\end{table}

In a similar fashion, the set of configurations that have been tried for the ES algorithm is a product of the parameters in Table~\ref{tab:ES_params}.
This time, we evaluate $8 \cdot 3 \cdot 4 \cdot 2 \cdot 2 \cdot 4 = 1536$ different configurations.
This experiment is twice as expensive as the one ran for the GA algorithm, which means it takes \textasciitilde $5$ hours on the same device.

\begin{table}[htbp]
    \centering
    \begin{tabular}{llllll}
        \toprule
        \texttt{N}, $\mu$, $\lambda$ & $\sigma$ & $\tau$ & ind. $\sigma$s & \texttt{chunk\_size} & \texttt{recombination}       \\
        \midrule
        100, 40, (60 \& 100)         & 0.01     & 0.1    & \textit{yes}         & 3                    & discrete               \\
        100, 20, (80 \& 100)         & 0.1      & 0.2    & \textit{no}          & 7                    & intermediate           \\
        40, 20, (20 \& 40)           & 0.5      & 0.5    & -                    & -                    & discrete (global)      \\
        40, 10, (30 \& 40)           & -        & 0.99   & -                    & -                    & intermediate (global)  \\
        \bottomrule
    \end{tabular}
    \vspace{0.1cm}
    \caption{The different parameters that have been tried for the ES algorithm.}
    \label{tab:ES_params}
\end{table}

Only when we have explored the behaviour of the algorithms with all these different configurations, will we be able to make a fair comparison between the two algorithms.
Once we have determined the ``optimal'' setting for each individual parameter, more detailed visualizations will explain the behaviour of the algorithms in more detail.
This more detailed analysis is done through the use of metrics such as AUC (Area Under the Curve) and ERT (Expected Running Time).
Visualizations for these experiments are done with help of the IOHanalyzer framework \cite{IOH}.

\section{Experimental Results}
\label{sec:results}

\subsection*{Search Space Exploration}
\label{sec:res_search}

\begin{wrapfigure}[30]{r}{0.65\textwidth}
    \vspace{-0.5cm}
    \centering
    \begin{subfigure}[b]{0.3\textwidth}
        \centering
        \includegraphics[width=\textwidth]{figs/scores_dist_only108.png}
        \caption{}
        \label{fig:search_dist}
    \end{subfigure}
    ~
    \begin{subfigure}[b]{0.3\textwidth}
        \centering 
        \includegraphics[width=\textwidth]{figs/scores_hist_only108.png}
        \caption{}
        \label{fig:search_hist}
    \end{subfigure}
    \vskip\baselineskip
    \begin{subfigure}[b]{0.3\textwidth}
        \centering
        \includegraphics[width=\textwidth]{figs/scores_dist_high_only108.png}
        \caption{}
        \label{fig:search_dist_high}
    \end{subfigure}
    ~
    \begin{subfigure}[b]{0.3\textwidth}
        \centering 
        \includegraphics[width=\textwidth]{figs/scores_hist_high_only108.png}
        \caption{}
        \label{fig:search_hist_high}
    \end{subfigure}
    \captionsetup{width=.55\textwidth}
    \caption{
        The distribution of scores for the search space exploration.
        \ref{fig:search_dist} \& \ref{fig:search_dist_high} show the raw number of occurrences for each observed score, and \ref{fig:search_hist} \& \ref{fig:search_hist_high} show these distributions in the form of a histogram with 100 bins.
    }
    \label{fig:search}
\end{wrapfigure}

The exhaustive search of the search space gives us insight into the attainability of the scores, and their distribution.
The results of this search are shown in Figure~\ref{fig:search}.

It is immediately apparent that the scores are not uniformly distributed, but instead are skewed towards higher scores.
While this may seem favorable at first, it is important to note that the scores are not normalized.
The second observation to be made is that there are only \textasciitilde $7.2 \cdot 10^7$ unique invalid solutions left in our pruned search space.
When we take into consideration that we discarded more than $60\%$ before our search, we can state that the number of invalid solutions left is negligible, and conclude that roughly $40\%$ of all solutions are valid.
The last result obtained from performing this search is that the absolute optimal score $f^*$ is \textasciitilde $0.95055$ and occurs $45$ times.

In subsequent sections, this optimal value will not appear in the results, as this would mean that all $20$ repetitions of a given configuration have found the optimal solution.
With a budget of $5000$, this is extremely unlikely, but there are still some configurations that come close to this score.

\newpage

\subsection*{Parameter Exploration}

In the plots below we show some of the results regarding the effect of using $\mu + \lambda$ versus $\mu, \lambda$ selection (Figure~\ref{fig:elitism}), for both the ES and GA algorithms.
Similarly, the effect of using different population sizes (Figure~\ref{fig:pop_size}) is shown for both optimizers, alongside a selection algorithm-specific parameters.
A figure compares two to four different parameter settings by splitting all configurations based on the specific parameter that is being examined.
Each subfigure consists of two parts.
On the foreground, a scatterplot is shown, where each point represents the $f^*$ of a single configuration of an algorithm, averaged over $20$ repetitions.
The configuration are ranked from worst to best along the x-axis based on their $f^*$, which is determines their position in the Pareto front.
The background of a sub-figure shows a histogram along the y-axis, to better demonstrate how the function values are distributed for the given parameter setting.
For a more comprehensive overview of all parameter exploration results, please refer to the Appendix (\ref{app:params_ga} \& \ref{app:params_es}).

\begin{figure*}[htbp]
    \centering
    \begin{subfigure}[b]{0.45\textwidth}
        \centering
        \includegraphics[width=\textwidth]{figs/GA/elitism_GA.png}
        \caption{}
        \label{fig:elitism_GA}
    \end{subfigure}
    ~
    \begin{subfigure}[b]{0.45\textwidth}
        \centering 
        \includegraphics[width=\textwidth]{figs/ES/elitism_ES.png}
        \caption{}
        \label{fig:elitism_ES}
    \end{subfigure}
    \captionsetup{width=.9\textwidth}
    \caption{
        Final function value comparison for all configurations with different elitism settings for the genetic algorithm (left) and evolution strategy (right).
        The background of each sub-figure shows a histogram along the y-axis, to better demonstrate how the function values are distributed for the given parameter setting.
    }
    \label{fig:elitism}
\end{figure*}

\begin{figure*}[htbp]
    \centering
    \begin{subfigure}[b]{0.45\textwidth}
        \centering
        \includegraphics[width=\textwidth]{figs/GA/pop_size_GA.png}
        \caption{}
        \label{fig:pop_size_GA}
    \end{subfigure}
    ~
    \begin{subfigure}[b]{0.45\textwidth}
        \centering 
        \includegraphics[width=\textwidth]{figs/ES/pop_size_ES.png}
        \caption{}
        \label{fig:pop_size_ES}
    \end{subfigure}
    \captionsetup{width=.9\textwidth}
    \caption{
        Final function value comparison for all configurations with different population sizes for the GA (left) and ES (right).
    }
    \label{fig:pop_size}
\end{figure*}

It seems that $\mu + \lambda$ selection is better than $\mu, \lambda$ selection for both the GA and ES algorithms, although the difference is not as pronounced for the GA algorithm.
This is surprising, as ES algorithms are known to generally perform better with $\mu, \lambda$ selection.
A possible explanation for this is that the mutation operator as implemented in our evolution strategy is not particularly effective for handling this discrete neural architecture search problem.

The population size is also an important parameter for both algorithms, as there is a direct trade-off between number of generations (i.e. number of times the selection, recombination and mutation operators are applied) and the population size (i.e. the diversity of solutions in the population).
The differences between GA and ES are still apparent, but between the two parameter settings, the differences are not as pronounced as for the elitism settings.
For both algorithms, the best results are obtained with a population size of $100$.

\newpage

Although they use different recombination methods, for both the GA and ES algorithms we have tested four different recombination settings, which are shown in Figure~\ref{fig:recombination}.
For our genetic algorithm, it can be observed that the best results are obtained with the uniform crossover operator.
The second best results are seen with $3$-point crossover, as it has a disproportionately high number of configurations that fall into the second-to-best bin.
The differences between all of these operators are very minor, so we conclude that the recombination operator is not the most important parameter for the GA algorithm.
The results for our evolution strategy are more interesting, as a clear divide between discrete and intermediate recombination operators can be observed.
Both the 2-parent and global variants of the discrete recombination operator perform better than their intermediate equivalents.
This too, is surprising, as the discrete recombination operator is usually considered to be less effective than the intermediate recombination operator in ESs \cite{earecombination}.

\begin{figure*}[htbp]
    \centering
    \begin{subfigure}[b]{0.45\textwidth}
        \centering
        \includegraphics[width=\textwidth]{figs/GA/recombination_GA.png}
        \caption{}
        \label{fig:recombination_GA}
    \end{subfigure}
    ~
    \begin{subfigure}[b]{0.45\textwidth}
        \centering 
        \includegraphics[width=\textwidth]{figs/ES/recombination_ES.png}
        \caption{}
        \label{fig:recombination_ES}
    \end{subfigure}
    \captionsetup{width=.9\textwidth}
    \caption{
        Final function value comparison for all configurations with different recombination operators for the GA (left) and ES (right) algorithm.
    }
    \label{fig:recombination}
\end{figure*}

\begin{wrapfigure}[12]{r}{.55\textwidth}
    \vspace{-2.5cm}
    \centering
    \includegraphics[width=.5\textwidth]{figs/GA/selection_GA.png}
    \captionsetup{width=.55\textwidth}
    \caption{
        Final function value comparison for all configurations with different selection operators for GA.
        }
        \label{fig:selection_GA}
\end{wrapfigure}

\vspace{2cm}
Performances of the four different selection operators that we have tested for the GA algorithm are shown in Figure~\ref{fig:selection_GA}.
Here, we do see some very distinct Pareto fronts, with the two best selection methods being \textit{tournament} and \textit{rank}.
\textit{Stochastic universal sampling} is the third best selection operator, and \textit{roulette wheel} is by far the worst.
This is in line with what was described in Section~\ref{subsec:ga_sel}.
Because the fitness scores are not normalized, the roulette wheel selection operator is not able to properly assign selection probabilities to the individuals in the population, whereas the other three operators are able to do so.

\newpage
    
\begin{wrapfigure}[21]{r}{.55\textwidth} 
    \vspace{-0.5cm}
    \centering
    \includegraphics[width=.5\textwidth]{figs/ES/tau_ES.png}
    \captionsetup{width=.55\textwidth}
    \caption{
        Final function value comparison for all configurations with different $\tau$ values for ES.
    }
    \label{fig:tau_ES}
\end{wrapfigure}

Our last set of experiments for the parameter exploration that are worth mentioning are the ones that we have done for the variables involved with the mutation of the ES algorithm.
As mentioned in the previous parts of this section, the mutation operator is crucial for the success of the ES algorithm.
The mutation operator is responsible for introducing diversity into the population, and it is therefore important that it is able to do so in a way that is beneficial for the search.
The mutation effect is controlled by the $\tau$ and $\sigma$ values. 
Figure \ref{fig:tau_ES} shows that a higher determined $\tau$, results in populations with less fit solutions overall. 
Mutating the solutions too much impacts the fitness score negatively.
Despite this, they all manage to find top performing solutions.
Comparably, tuning the $\sigma$ has consequences to the performance.
In figure \ref{fig:sigma_ES} it can be seen that a $\sigma$ of 0.1 performs slightly better than 0.01, and just as good as 0.5. 
Considering that $\tau$ and $\sigma$ both influence the mutation rate, no definitive conclusion can be made from isolated tests. 

\vspace{5cm}

\begin{figure*}[htbp]
    \centering
    \includegraphics[width=\textwidth]{figs/ES/sigma_ES.png}
    \captionsetup{width=.55\textwidth}
    \caption{
        Final function value comparison for all configurations with different $\sigma$ values for ES.
    }
    \label{fig:sigma_ES}
\end{figure*}

\newpage

\subsection*{Final Configurations}
\label{subsec:final_configs}

Having completed the experiments with the various configurations of both ES and GA, we have defined the final fittest configurations for the algorithms.
After inspecting all varied parameter's performances individually, we compiled a set of four configurations for GA (Table~\ref{tab:GA_final_configs}) and 4 configurations for ES (Table~\ref{tab:ES_final_configs}).

\begin{table}[htbp]
    \centering
    \begin{tabular}{c|cccc|cc}
        \toprule
        \texttt{config} &
        \texttt{N}$^\dagger$, $\mu$, $\lambda$ & \texttt{selection} & \texttt{recombination}$^\dagger$ & \texttt{mutation}
            & $f^*_{\mathrm{avg}} $ & $AUC$ \\ % this is still the header of the table
        \midrule
        1 & 100, 80, 20  &  tournament  &  uniform   &  uniform (0.04)  &  \textbf{0.9478}  &   0.9871           \\
        2 & 100, 80, 20  &  rank        &  uniform   &  uniform (0.01)  &  0.9458           &   0.9863           \\
        3 & 100, 90, 10  &  tournament  &  uniform   &  uniform (0.04)  &  0.9462           &   \textbf{0.9928}  \\
        4 & 100, 90, 10  &  rank        &  uniform   &  uniform (0.01)  &  0.9456           &   0.9862           \\
        \bottomrule
    \end{tabular}
    \vspace{0.1cm}
    \captionsetup{width=.75\textwidth, justification=centering}
    \caption{
        The four configurations of the genetic algorithm used for the final experiments.
        Parameters noted with $^\dagger$ are fixed to reduce complexity.
    }
    \label{tab:GA_final_configs}
\end{table}

\begin{table}[htbp]
    \centering
    \begin{tabular}{c|cccccc|cc}
        \toprule
        \texttt{config} &
        \texttt{N}$^\dagger$, $\mu$, $\lambda$ & $\sigma$ & $\tau^\dagger$ & ind. $\sigma$s$^\dagger$ & \texttt{chunk\_size}$^\dagger$ & \texttt{recombination}
            & $f^*_{\mathrm{avg}}$ & $AUC$ \\ % this is still the header of the table
        \midrule
        1 & 100, 80, 20  &  0.5  &  0.5  &  \textit{yes}  &  3  & discrete           &  0.9495           &  0.9887         \\
        2 & 100, 80, 20  &  0.1  &  0.5  &  \textit{yes}  &  3  & discrete (global)  &  \textbf{0.9501}  &  0.9919         \\
        3 & 100, 90, 10  &  0.5  &  0.5  &  \textit{yes}  &  3  & discrete           &  0.9495           &  0.9919         \\
        4 & 100, 90, 10  &  0.1  &  0.5  &  \textit{yes}  &  3  & discrete (global)  &  0.9488           & \textbf{0.9944} \\
        \bottomrule
    \end{tabular}
    \vspace{0.1cm}
    \captionsetup{width=.75\textwidth, justification=centering}
    \caption{
        The four configurations of the evolution strategy used for the final experiments.
        Parameters noted with $^\dagger$ are fixed to reduce complexity.
    }
    \label{tab:ES_final_configs}
\end{table}

The final plots originate from IOH analyzer in order to measure the performance of the GA and ES algorithm. 
The AUC values from the ECDF curves of both algorithms have already been shown, but the plots themselves are displayed here. 
Figure \ref{fig:GA_ECDF} shows the proportion of runs that have found a solution with a target value of 0.94 ($f_{\mathrm{target}})$. 
The x-axis show the amount of function evaluations needed to succeed reaching $f_{\mathrm{target}}$
It can be seen that the 3rd configuration needs the least amount of function evaluations, followed by configuration 2, 4 and 1. 
The ES ECDF figure (\ref{fig:ES_ECDF}) shows configuration 4 performing best followed by 2 and 3 simultaneously, and 1 finishing last.

\begin{figure*}[htbp]
    \centering
    \includegraphics[width=.75\textwidth]{figs/IOH/GA_ECDF.png}
    \captionsetup{width=.75\textwidth}
    \caption{ECDF curves for the four GA configurations measured earlier.}
    \label{fig:GA_ECDF}
\end{figure*}

\newpage

\begin{figure*}[htbp]
    \centering
    \includegraphics[width=.85\textwidth]{figs/IOH/ES_ECDF.png}
    \captionsetup{width=.85\textwidth}
    \caption{ECDF curves for the four ES configurations measured earlier.}
    \label{fig:ES_ECDF}
\end{figure*}

\begin{wrapfigure}[25]{r}{0.65\textwidth}
    \vspace{-1cm}
    \centering
    \includegraphics[width=.65\textwidth]{figs/IOH/GA_ERT.png}
    \captionsetup{width=.55\textwidth}
    \caption{The result of IOH analyzer Expected Run Time (ERT) analysis on the four GA configurations measured earlier.}
    \label{fig:GA_ERT}
\end{wrapfigure}

\vspace{1.8cm}

The other performance measure were the ERT results of both GA and ES. 
Figure \ref{fig:GA_ERT} shows to the amount of evaluations needed to find a solution reaching a fitness of 0.94 using GA. 
Here we can see that configuration 2, 3 and 4 need the most amount of evaluations, and configuration 1 is the better performing one.

Figure \ref{fig:ES_ERT} shows the ERT curves of the ES algorithm. 
With small margins, it is noticeable that configuration 4 required the most evaluations, followed by configuration 1 and 3 respectively.
Making configuration 2 the better performing algorithm.

The results from the AUC are related to the performance of each configuration.
An increased AUC means that fewer function evaluations have to be made before reaching $f_{\mathrm{target}}$.
A relation between $f^{*}_{\mathrm{avg}}$ and ERT performance can also be seen.
Where a high $f^{*}_{\mathrm{avg}}$ means the the ERT performance is elevated, thus less evaluations are needed.

\newpage

\begin{figure*}[htbp]
    \centering
    \includegraphics[width=.65\textwidth]{figs/IOH/ES_ERT.png}
    \captionsetup{width=.65\textwidth}
    \caption{The result of IOH analyzer Expected Run Time (ERT) analysis on the four ES configurations measured earlier.}
    \label{fig:ES_ERT}
\end{figure*}

Figure \ref{fig:GA_f1} presents the fixed-budget results of the 4 final GA configurations.
This plot shows that the algorithm is characterized by a strong start, all runs containing at least one solution with a fitness score between $0.940$ \& $0.945$.
After this substantial beginning however, a semi stagnant stage is hit, where the rate of improvement is low.
There are some outliers; the thin lines reaching the theoretical limit (dashed gray line), are some of the fittest possible solutions to be found.
The red downwards spike in generation 11 shows that it is possible to end up with a worse solution.
Overall, the GA shows a robust performance, having more positive outlier spikes than negative.

\begin{figure*}[htbp]
    \centering
    \includegraphics[width=.65\textwidth]{figs/GA/GA_f1.png}
    \captionsetup{width=.7\textwidth}
    \caption{
        Best fitness found per population for the final four configurations of GA.
        Fixed parameters are not reflected in the labels, but can be found in Table~\ref{tab:GA_final_configs}.
    }
    \label{fig:GA_f1}
\end{figure*}

\newpage

The previous figure showed the average performance, where no clear increase of performance is seen.
When we only consider the best solution found in a run so far (see Figure~\ref{fig:GA_f2}), an apparent gain of fitness can be recognized more easily.
Final configuration 4 of the GA performs the best, which was also determined by the AUC analysis and Figure~\ref{fig:GA_f1}.

\begin{figure*}[htbp]
    \centering
    \includegraphics[width=.75\textwidth]{figs/GA/GA_f2.png}
    \captionsetup{width=.75\textwidth}
    \caption{Best fitness found so far (over the course of the optimization process) for the final four configurations of GA.}
    \label{fig:GA_f2}
\end{figure*}

The fixed-budget results for the final four configurations of ES are shown in Figure~\ref{fig:ES_f}.
Please note that the standard deviation reaches higher than the theoretical limit in some cases, and that this is okay, because these values do not reflect actual observations.
The first thing that appears is that the best fitness found per population is strictly increasing.
This was not the case for the genetic algorithm, where -- while trending upwards -- drops in fitness were observed regularly.
While both algorithms employ \textit{elitist} selection (i.e., the best $\mu$ solutions are always carried over to the next generation), GA uses stochastic parent selection, while ES is purely deterministic in this regard.
Therefore, in the case of GA, there exists the possibility for a population $\mathbf{X}_t$ to have a lower $f^*_{\mathrm{found}}$ (highest observed fitness) than the previous population $\mathbf{X}_{t-1}$, whereas this is not possible for ES.

\begin{figure*}[htbp]
    \centering
    \includegraphics[width=.75\textwidth]{figs/ES/ES_f.png}
    \captionsetup{width=.75\textwidth}
    \caption{
        Best fitness found per population for the final four configurations of ES.
        Fixed parameters are not reflected in the labels, but can be found in Table~\ref{tab:ES_final_configs}.
    }
    \label{fig:ES_f}
\end{figure*}

Secondly, the number of individual runs (thin lines) of the algorithm that reach the optimal fitness $f^*$ is significantly higher for ES than for GA.
Last, but certainly not least, the averaged $f^*_{\mathrm{found}}$ (thick line) when the budget is exhausted is considerably higher, mainly due to a much faster convergence at the start of the run.
To illustrate this, Figure~\ref{fig:ES_f} compares the best performing\footnote{
    Here, the best performing configuration is defined as the configuration that has the highest average $f^*_{\mathrm{found}}$ when the budget is exhausted out of the four configurations that were selected for this analysis.
} configurations of GA and ES.
While not concrete evidence of superiority, it is at the very least interesting to see that our adaptation of an evolutionary strategy on a discrete problem is by no means inferior to a genetic algorithm.

\begin{figure*}[htbp]
    \centering
    \includegraphics[width=.75\textwidth]{figs/GA_vs_ES.png}
    \captionsetup{width=.75\textwidth}
    \caption{
        Comparison of the best performing (out of four empirically selected) configurations of GA versus ES.
    }
    \label{fig:GA_vs_ES}
\end{figure*}

The fact that ES seems to be better suited for this problem is not surprising, as this was already foreshadowed in the parameter exploration experiments.
While we only examined the parameters with the distribution plots, the same thing can be done for all of the configurations for GA versus ES, as shown in Figure~\ref{fig:GA_vs_ES_exp}.
Using these observations, we can more confidently state that the evolutionary strategy approach is the more promising one.

\begin{figure*}[htbp]
    \centering
    \includegraphics[width=.75\textwidth]{figs/GA_vs_ES_exp.png}
    \captionsetup{width=.75\textwidth}
    \caption{
        Comparison of all configurations of GA versus ES, shown as a distribution of the $f^*_{\mathrm{avg}}$ when the budget is exhausted.
    }
    \label{fig:GA_vs_ES_exp}
\end{figure*}

\newpage

\section{Discussion and Conclusion}
\label{sec:disres}

The results of this study show that the evolutionary strategy approach is more promising than the genetic algorithm approach.
This is not what we initially expected, as we were under the impression that the genetic algorithm would be the better choice, due to its discrete nature.
However, the results show that the evolutionary strategy approach is more promising, as it is able to find the optimal solution faster and more reliably.
This is not to say that the genetic algorithm approach is not useful, as it can still be used to find a good solution, but from the results it can be stated that it is slower to converge when compared to the evolutionary strategy approach.
The results also show that the evolutionary strategy approach is more robust to changes in the parameters, as the results are more consistent across the different configurations.

\newpage

\bibliographystyle{unsrt}  
\bibliography{references}  


\newpage



% ---------------------------------------------------------------------------------------------------------------------------
%                                                      V dont touch V
% ---------------------------------------------------------------------------------------------------------------------------



\appendix
\section{Appendix}
\label{sec:app}


\subsection{Parameter Plots - Genetic Algorithm}
\label{app:params_ga}

\begin{figure*}[htbp]
    \centering
    \includegraphics[width=\textwidth]{figs/GA/pop_size_GA.png}
    \caption{The effect of population size on accuracy performance.}
    \label{fig:A_GA_popsize}
\end{figure*}

\begin{figure*}[htbp]
    \centering
    \includegraphics[width=\textwidth]{figs/GA/elitism_GA.png}
    \caption{The effect of elitism on accuracy performance.}
    \label{fig:A_GA_elitism}
\end{figure*}

\newpage

\begin{figure*}[htbp]
    \centering
    \includegraphics[width=\textwidth]{figs/GA/selection_GA.png}
    \caption{The effect of selection operator on accuracy performance.}
    \label{fig:A_GA_selection}
\end{figure*}

\newpage

\begin{figure*}[htbp]
    \centering
    \includegraphics[width=\textwidth]{figs/GA/recombination_GA.png}
    \caption{The effect of using different recombination operators on accuracy performance.}
    \label{fig:A_GA_recombination}
\end{figure*}

\newpage

\begin{figure*}[htbp]
    \centering
    \includegraphics[width=.85\textwidth]{figs/GA/mutation_GA.png}
    \caption{The effect of using different mutation operators on accuracy performance.}
    \label{fig:A_GA_mutation}
\end{figure*}

\begin{figure*}[htbp]
    \centering
    \includegraphics[width=.85\textwidth]{figs/GA/bit_flips_GA.png}
    \caption{The effect of number of bit flips on accuracy performance for the bitflip mutation operator.}
    \label{fig:A_GA_bitflip}
\end{figure*}

\begin{figure*}[htbp]
    \centering
    \includegraphics[width=.85\textwidth]{figs/GA/mutation_rate_GA.png}
    \caption{The effect of mutation rate on accuracy performance for the uniform mutation operator.}
    \label{fig:GA_mutation_rate}
\end{figure*}

\newpage

\subsection{Parameter Plots - Evolution Strategy}
\label{app:params_es}

\begin{figure*}[htbp]
    \centering
    \includegraphics[width=\textwidth]{figs/ES/pop_size_ES.png}
    \caption{The effect of population size on accuracy performance.}
    \label{fig:A_ES_popsize}
\end{figure*}

\begin{figure*}[htbp]
    \centering
    \includegraphics[width=\textwidth]{figs/ES/elitism_ES.png}
    \caption{The effect of elitism on accuracy performance.}
    \label{fig:A_ES_elitism}
\end{figure*}

\newpage

\begin{figure*}[htbp]
    \centering
    \includegraphics[width=.65\textwidth]{figs/ES/chunk_size_ES.png}
    \caption{The effect of chunk size on accuracy performance.}
    \label{fig:A_ES_chunksize}
\end{figure*}

\begin{figure*}[htbp]
    \centering
    \includegraphics[width=.9\textwidth]{figs/ES/sigma_ES.png}
    \caption{The effect of using different sigmas on accuracy performance.}
    \label{fig:A_ES_sigmas}
\end{figure*}

\begin{figure*}[htbp]
    \centering
    \includegraphics[width=.65\textwidth]{figs/ES/individual_sigmas_ES.png}
    \captionsetup{width=0.7\textwidth}
    \caption{
    The effect of using individual sigmas for each gene, versus two sigmas for every individual (one for the matrix part, and one for the operations part).
    }
    \label{fig:A_ES_indivsigma}
\end{figure*}

\newpage

\begin{figure*}[htbp]
    \centering
    \includegraphics[width=\textwidth]{figs/ES/recombination_ES.png}
    \caption{The effect of using different recombination operators on accuracy performance.}
    \label{fig:A_ES_recombination}
\end{figure*}

\newpage

\begin{figure*}[htbp]
    \centering
    \includegraphics[width=\textwidth]{figs/ES/tau_ES.png}
    \caption{The effect of using different tau rates on accuracy performance.}
    \label{fig:A_ES_tau}
\end{figure*}

\newpage

\subsection{Workflow python scripts}
\label{app:workflow}

\begin{figure}[htbp]
    \centering
    \includegraphics[width=\textwidth]{figs/workflow_EA.drawio (1).png}
    \caption{ 
        This workflow shows the general interactions of the different functions. 
        Hopefully helping to create a better sense of connectivity between the different seen operations. 
        The different coloring represent the connection to the file they belong in; 
        \texttt{main.py} (blue), \texttt{evolution\_strategies.py} (green), \texttt{genetic\_algorithm.py} (red) and purple being function needed to call the script from other sources.
    }
    \label{fig:A_workflow}
\end{figure}

\end{document}
